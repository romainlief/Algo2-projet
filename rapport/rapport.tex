\documentclass[utf8]{article}

\usepackage[utf8]{inputenc}

\usepackage[parfill]{parskip}

\usepackage{amsmath}
\usepackage{amssymb}
\usepackage{amsfonts}
\usepackage{graphicx}
\usepackage{float}
\usepackage{listingsutf8}

\usepackage{fullpage}
\usepackage{hyperref}
\usepackage{url}
\usepackage{color}
\usepackage{listings}
\usepackage{caption}
\usepackage{subcaption}
\usepackage{enumitem}

% -----------------------------------------------------

\begin{document}

\begin{titlepage}
    \centering
    
    % Titre en haut de la page
    \vspace*{1cm}
    {\huge \bfseries INFO F-203\\
                    Rapport \par}
    
    % Espace vertical pour centrer le logo
    \vfill
    
    % Logo au milieu de la page
    \begin{figure}[h]
        \centering
        \includegraphics[scale=0.2]{logo.png}
    \end{figure}
    
    % Espace vertical pour descendre l'auteur et la date en bas
    \vfill
    
    % Auteur et date en bas de la page
    {\large Auteurs: Liefferinckx Romain, Rocca Manuel\\ 
            Matricules: 000591790, 000596086\\ 
            Section: INFO \par}
    {\large \today \par}
\end{titlepage}

\newpage
\tableofcontents

\newpage

% -----------------------------------------------------

\section{Introduction}
Dans le cadre de notre cours d'algorithmique INFO F-203, l'occasion s'est présentée à nous de créer un programme cherchant un chemin optimal
entre un point A et un point B sur base d'une heure de départ. En effet, sur base d'un ensemble de données
fournies sous format \emph{General Transit Feed Specification} (\textbf{GTFS}), nous avons utilisé le \emph{Connexion Scan Algorithm} (\textbf{CSA})
pour implémenter notre chercheur de chemin en \emph{Java}. 

Dans les sections à suivre, nous abordons le parsing des données, des structures formées à partir de celles-ci pour notre
implémentation et certains détails techniques comme la complexité temporelle et spatiale. Nous justifierons également certains choix
comme celui de l'algorithme précisé ci-dessus, à savoir le \textbf{CSA}.


\section{Parsing}
Dans cette section nous expliquons les procédés utilisés pour charger les données en mémoire à partir des fichiers CSV fournis ainsi que
les structures de données utilisées pour leur stockage et leur utilisation optimale dans l'algorithme choisi par nos soins.

\subsection{Structures de données pour le stockage}
Dû à notre choix d'implémentation algorithmique, nous avons opté pour des structures efficaces pour utiliser l'utiliser dans
les meilleurs conditions possibles. En effet, le CSA, comme son nom le suggère, fait une forte utilisation des connexions entre arrêts,
chose que nous détaillons plus loin dans ce rapport.

D'abord ont été créées les quatre classes principales, chacune correspondant à un type de fichier CSV donné. De manière générale, 
chaque champ de chaque fichier est repris comme un attribut de classe. Cela n'est cependant pas toujours le cas, nous le précisons
dans les sections adéquates.

\subsubsection{La classe Route}
Cette classe est une simple classe de stockage, chaque attribut correspondant à un champ des fichiers \emph{routes.csv}.

% Table qui détaille la classe Route
\begin{table}[h]
    \centering
    \begin{tabular}{|l|l|p{8cm}|}
    \hline
    \textbf{Attribut} & \textbf{Type} & \textbf{Description} \\
    \hline
    routeId & final String & L'id de la route représentée \\
    routeShortName & final String & Le nom de la route raccourci \\
    routeLongName & final String & Le nom de la route complet \\
    routeType & final String & Le type de véhicule utilisant cette route \\
    \hline
    \end{tabular}
    \caption{Classe Route}
\end{table}

\subsubsection{La classe StopTime}
La particularité de cette classe est qu'il lui manque le champ tripId donné dans les CSV concernés. Ce choix découle
de la structure de la classe Trip détaillée dans la section 2.1.4. % TODO: ajouter un label au tableau trip ici
\vspace{2cm}
% Table qui détaille la classe StopTime
\begin{table}[h]
    \centering
    \begin{tabular}{|l|l|p{8cm}|}
    \hline
    \textbf{Attribut} & \textbf{Type} & \textbf{Description} \\
    \hline
    departureTime & final String & L'heure du départ à partir de
                                   l'arrêt associé sur le trajet associé en format 
                                   heure;minutes;secondes \\
    stopId & final String & L'id de l'arrêt associé \\
    stopSequence & final int & Le numéro de l'arrêt dans le trajet associé \\
    \hline
    \end{tabular}
    \caption{Classe StopTime}
\end{table}

\subsubsection{La classe Stop}
Dans cette classe, hormis le fait que chaque champ des fichiers \emph{routes.csv} est repris, nous avons fait le choix d'ajouter
une liste de tripId, permettant de retrouver efficacement chaque Trip partant de cet arrêt. 

% Table qui détaille la classe Stop
\begin{table}[h]
    \centering
    \begin{tabular}{|l|l|p{8cm}|}
    \hline
    \textbf{Attribut} & \textbf{Type} & \textbf{Description} \\
    \hline
    stopId & final String & L'id du stop représenté \\
    stopName & final String & Le nom du stop \\
    stopLat & final String & La latitude du stop \\
    stopLon & final String & La longitude du stop \\
    trip\_ids & List$<$String$>$ & La liste des tous les trips (leurs ids) partant de ce stop \\
    \hline
    \end{tabular}
    \caption{Classe Stop}
\end{table}

\subsubsection{La classe Trip}
Une fois de plus, en plus des champs trouvés dans les fichiers \emph{trips.csv} retranscrits en attribut, nous avons ajouté
une liste de StopTime associés à ce Trip. Ceci nous permet de construire efficacement les connexions entre arrêts affublées
de temps de départ et d'arrivée.

% Table qui détaille la classe Trip
\begin{table}[h]
    \centering
    \begin{tabular}{|l|l|p{8cm}|}
    \hline
    \textbf{Attribut} & \textbf{Type} & \textbf{Description} \\
    \hline
    tripId & final String & L'id du trip représenté \\
    routeId & final String & L'id de la route sur laquelle il passe \\
    stopTimes & List$<$StopTime$>$ &  La liste de tous les StopTime associés à ce trip \\
    \hline
    \end{tabular}
    \caption{Classe Trip}
\end{table}

\section{Le Connexion Scan Algorithm}

\section{Sources-bibliographie}

\end{document}